%%%%%%%%%%%%%%%%%%%%%%%%%%%%%%%%%%%%%%%%%%%%%%%%%%%%%%%%%%%%%%%%%%%%%
%
%   Statistical Methods for Analysis With Missing Data 
%
%   Syllabus
%
%   Last modified 01/12/2026-- Accessible version
%
%%%%%%%%%%%%%%%%%%%%%%%%%%%%%%%%%%%%%%%%%%%%%%%%%%%%%%%%%%%%%%%%%%%%%%

%%  Document meta data  -- needed for tagging, etc - must be located
%%  BEFORE \documentclass -- see this page at Texas A&M
%%  https://esail.tamu.edu/faculty-tutorials/accessible-latex-pdf-ua-2-overleaf-2025/

\DocumentMetadata{
  tagging = on,
lang = en,
pdfstandard = ua-2,
  pdfstandard = a-4f, % optional archival standard
  tagging-setup = {math/setup={mathml-AF,mathml-SE},
                   extra-modules={verbatim-mo},
                   table/header-rows=1}}

\documentclass[11pt]{article}

\usepackage{graphicx}
\usepackage{helvet}
\usepackage{amsmath,amsfonts}
\usepackage{amssymb}

\usepackage{unicode-math}
\usepackage{hyperref}

\usepackage{sectsty}
\allsectionsfont{\normalfont\sffamily\bfseries}

%
\setlength{\oddsidemargin}{0.0in}
%\setlength{\evensidemargin}{0.0in}
\setlength{\topmargin}{0.0in}
\setlength{\headheight}{0.0in}
\setlength{\headsep}{0.0in}
% \renewcommand{\headrulewidth}{0.2pt}
% \renewcommand{\footrulewidth}{.15pt}
%\setlength{\evensidemargin}{0.0in}
\setlength{\evensidemargin}{-0.25in}
\setlength{\headheight}{.20in}
\setlength{\topmargin}{-.75in}
\setlength{\headsep}{3ex}
\setlength{\parskip}{2.3ex}
%\setlength{\footskip}{2ex}
\setlength{\baselineskip}{2ex}
\setlength{\parindent}{0pt}
%\renewcommand{\baselinestretch}{1}

%
\setlength{\textheight}{9.6in}
\setlength{\textwidth}{6.7in}

\def \doublespace{\openup 2\jot}
\renewcommand{\baselinestretch}{1.0}
\tolerance=500

%%  This will set the document properties -- hopefully it sets the
%%  Document title as claimed

\hypersetup{ 
  pdftitle={Statistical Methods for Analysis With Missing Data
    Reference List}, 
  pdfauthor={davidian@ncsu.edu}, 
  pdfsubject={Course notes}, 
  pdfkeywords={LaTeX, accessibility, PDF/UA-2, NVDA, MathML}, 
  pdfdisplaydoctitle=true 
}
    
\begin{document}

\sf

\pagestyle{empty}

\newcommand{\emf}{\sf \em}
\newcommand{\embf}{\sf \textbf}
\newcommand{\bfem}[1]{\textbf{\em #1}}

\begin{Large}
\begin{center}
\textbf{ST 790-003, Statistical Methods for Analysis With Missing Data\\ Spring 2026} 
\end{center}
\end{Large}

\pdfbookmark[1]{Course Description}{s:description}
\section*{Course Description}

Missing data are ubiquitous in almost every area of
scientific inquiry, and especially in health sciences research
involving human subjects.

The classical definition of missingness is that data that were
intended to be collected in a prospective study were not. For example,
in a medical study designed to collect data longitudinally on every
participant at a prespecified series of follow-up times, some subjects
may fail to appear at the clinic for intended measurements at one or
more times or, more ominously, drop out of the study and never return
after a certain point. If the reasons for failure to appear or dropout
are related to the issues under study, e.g., if subjects who are
benefiting less from their assigned treatments are more likely to drop
out, intuitively, failure to acknowledge this somehow could distort
conclusions. Missing data of this type are such a great challenge in
pharmaceutical and biotechnology research that in 2010 the US Food and
Drug Administration asked the National Research Council of the
National Academy of Sciences to convene an expert Panel on the
Handling of Missing Data Clinical Trials to develop guidance on how
missingness should be handled in the regulatory context.

Missingness also occurs in data that arise in other contexts. For
example, in retrospective analysis of observational data already
collected, such as those from completed studies or captured in other
large databases, it is routinely the case that not all variables are
available on all subjects or other units. Contrary to popular belief,
``big data'' are not somehow exempt from issues of missingness. For
example, the use of electronic health records, which are of course
observational in nature, is of great current interest in the study of
the comparative effectiveness of drugs and other interventions. The
fact that some subjects have relatively more observations than others
often reflects the fact that less healthy individuals tend to have
more encounters with the healthcare system. These same individuals may
also be more likely to receive certain interventions and to have
health outcomes worse than those those of healthier individuals. Thus,
there is a relative ``missingness'' of information among different
individuals that is likely associated with the issues under study.

Missing data have important implications for analysis. At the very
least, there is a loss of information and reduction in precision of
inference on the population of interest relative to that intended. Of
much greater concern is the potential for biased and misleading
inferences that can result if the reasons for missingness are related
to outcomes of interest. Accordingly, principled methods to take this
challenge into appropriate account are required.

This course provides an overview of modern statistical frameworks and
methods for analysis in the presence of missing data. Both
methodological developments and applications are emphasized. 

\pdfbookmark[1]{Logistics}{s:logistics}
\section*{Logistics}

\bfem{Lectures:}  T-Th, 3:00 - 4:15 pm, 5270 SAS Hall (3 credit hours)

\bfem{Course website:}    \url{https://www4.stat.ncsu.edu/~davidian/st790/}

 \bfem{Prerequisites:}  ST 702, ST 704, ST 705 (or equivalents at the
   PhD level); basic knowledge of statistical models, programming skills

\bfem{Instructor:}  \textbf{Marie Davidian}, 5124 SAS Hall,
davidian@ncsu.edu, \url{www4.stat.ncsu.edu/~davidian}

\bfem{Office hours:} On Zoom Th, 1:00 - 2:30 pm, and by appointment
(Zoom link available on the course website)

%\bfem{Teaching Assistant:} 
%  \textbf{Peter Norwood},
%  pnorwoo@ncsu.edu, Zoom office hours M 7:00 - 8:00 pm

\pdfbookmark[1]{Learning Outcomes}{s:goals}
\section*{Learning Outcomes}

Upon completion of the course, students will be able to:
\begin{enumerate}
\item Draw on a foundation of knowledge regarding the main fundamental
  frameworks and methods for handling missing data that will enable 
  them to read the current literature and undertake research in this
  area

\item Appreciate of the implications of missing data for valid
  inference and understand clearly the pros and cons of the major
  classes of popular methods for handling missing data

\item Select appropriate methods in practice, explain the assumptions
  required for their principled use, and implement the methods
  themselves or using available software.
\end{enumerate}

\pdfbookmark[1]{Course Test}{s:text}
\section*{Course Text}

Lecture notes prepared by the instructor.  These notes are available
on the course website.

\pdfbookmark[1]{Course Delivery}{s:delivery}
\section*{Course Delivery}

All lectures for the course will be delivered in real time and in
person durin scheduled class meetings (synchronous
delivery). 

\bfem{Class recordings:} The instructor will record lectures and make them available upon
request to registered students.  The recordings are meant as a resource for
students to revisit the material rather than as an alternative to
attendance in real time. The recordings will also allow students who
must miss a lecture due to illness or other conflict to review the
material they missed.  Recordings will capture the entire lecture and
will include questions and comments of both instructor and students.
These recordings may be used beyond the current semester or in any
setting outside of the course.  Contact the instructor if you have
concerns.

\pdfbookmark[1]{Attendance}{s:attend}
\section*{Attendance}

Regular attendance of lectures is expected.  If you must miss a
lecture due to illness, job interview, etc, please inform the
instructor in advance of the lecture by email.

Chronic unexcused absenteeism will result in at least a 5 point
reduction in the Final Score, as determined by the instructor.

\pdfbookmark[1]{Communication}{s:comm}
\section*{Communication}

Office hours for the instructor will be held online on Zoom.  Students
can email the instructor anytime to make an appointment to meet
outside of office hours on Zoom to ask questions about course material
or homework assignments.

Other than Zoom office hours, the primary mode of communication
between the instructor and students will be email.  Please use your NC
State email and the subject line ``ST 790.''  I should respond within
one day.

Students should check their email often for announcements from the
instructor, and students should feel free to email the instructor with
questions or concerns.  Please use your NC State email and the subject
line ``ST 790.''

\pdfbookmark[1]{Grading}{s:grading}
\section*{Grading}

Final grade will be determined by the Final Score =
0.75$\times$H +  0.25$\times$F, where 
H is the average on homework and 
F is the score on the final project, where each is scored out of 100. 
{\em There will no midterm or final exam}.  Conversion of this score
into a letter grade will be made according to the following tentative
grading scale (the upper score in each range except A+ belongs to the
next highest grade): A+, 100; A, 92-99; A-, 90-92; B+, 88-90; B 82-88;
B-, 78-82.  Scores below 70 will be handled on a case-by-case basis.

Auditors must attend class and attempt and turn in homework problems.
Auditors are not required to complete the final project.

\pdfbookmark[1]{Homework}{s:homework}
\section*{Homework}

There will be four (4) homework assignments.  Each homework assignment
will comprise both analytical problems and data analysis problems.
The analytical problems will involve derivations, proofs, and
simulation studies.  The data analysis problems will involve carrying
out analyses of data using the methods discussed in the lectures as
programmed by you or implemented in available software.  For problems
involving programming, the relevant parts of the program and its
output should be turned in, along with interpretation of the results
as dictated by the problem. 

Students are permitted and even encouraged to work together on
homework; however, each student must prepare his/her own solutions.
Blind copying of the work of other students demonstrates that the
student doing the copying is not serious about developing the
independence required for a PhD and has obvious disadvantages for
mastery of the material.

Homework assignments should be typed and prepared as pdf files.
Completed homeworks should be emailed to the instructor {\em at or
  prior to the beginning of class} on the due date.

\bfem{Use of artificial intelligence and other resources:} Use of
any resources beyond the course materials is permitted (books,
articles, artificial intelligence (AI), etc) is permitted in
completing the homework assignments.  However, students must
synthesize the information from such sources in preparing and writing
up his/her own solutions {\em without AI assistance}.

\bfem{Originality-checking software:} The instructor will not use
originality-checking software to detect originality of student
homework submissions.  Rather, the instructor expects students to
follow the above requirement, as, again, blind reliance on such
resources has obvious disadvantages for mastery of the material.

\bfem{Late homework:} Unexcused late homework will be discounted by
50\%.  If you anticipate having problems meeting the due date, please
contact the instructor by email at least one day prior to the due
date.

\bfem{Tentative assignments/due dates:}  Definitive information
will be posted on the course website.  

\begin{tabular}{ll}
Homework 1 &  Chapters 1-2, due Tuesday, February 3\\
Homework 2 & Chapter 3, due Tuesday, March 3\\
Homework 3 & Chapters 4-5, due Thursday, April 2\\
Homework 4 & Chapters 5-7, due Thursday, April 23\\
\end{tabular}

\pdfbookmark[1]{Final Project}{s:project}
\section*{Final Project}

A final project will be assigned in which teams of students will
collaborate on identifying a paper in the recent literature (2020 or
later) that reports on methodology that goes beyond the foundational
material discussed in class.  Each team will prepare a set of slides
(no more than 20) presenting a summary of the key methodology proposed
in the paper, using the notation developed in this course (to the
extent possible), along with a discussion of how the methodology is
related to work reviewed in the course and anything else the team
finds relevant. The slides should make the main, big ideas and
advances the paper accessible to anyone with the same background in
this area gained from the course (e.g., fellow students). Thus, the
emphasis should be on the big picture and not on technical details.

Teams will turn in their slides for evaluation by and feedback from
the instructor.  Tentatively, teams will also report their work in an oral
presentation to the class (size of class permitting), possibly held on
the last day of class, during the scheduled final exam slot for the
course, or at another mutually agreeable time during the exam period
(to be determined later in the semester by class consensus).

Teams will be determined at random by the instructor.  

\pdfbookmark[1]{Computing/Software}{s:computing}
\section*{Computing/Software}

Implementation of methods discussed in the lectures in SAS and R
software will be demonstrated.  Students are free to program in the
language of their choice and can use any software of their choice,
including but not limited to SAS and R.  Access to computing resources
that support SAS and R is thus required.

\pdfbookmark[1]{Tentative Schedule}{s:schedule}
\section*{Tentative Schedule}

Tentative content of course meetings, January 13 - April 28, 2026; Wellness Day
is February 17, Spring Break is March 16-20.  Final Exam slot is May
5, 3:30 - 6:00 pm.  Dates of coverage may be subject to change.
\begin{description}
\item 01/13 - 01/20 -- Introduction and Motivation
\begin{itemize}
\item Fundamental problem of missing data
 \item Examples
\item Statistical framework and taxonomy of missing data mechanisms
\item Review of estimating equations
\end{itemize}

\item 01/20 - 01/22 -- Na{\"{i}}ve Methods
\begin{itemize}
\item Complete or available case methods
\item Single imputation methods
\item Last Observation Carried Forward (LOCF)
\end{itemize}

\item 01/27 - 02/12 -- Likelihood-based Methods Under Missing At
  Random (MAR)
\begin{itemize}
\item Review of maximum likelihood inference for full data
\item Factorization of the density of $(R,Z)$
\item Observed data likelihood and ignorability
\item Expectation-Maximization (EM) algorithm
\item Missing information principle
\item Inference in practice
\item Bayesian inference
\end{itemize}

\item 02/12 - 03/10 -- Multiple Imputation Methods Under Missing At
  Random (MAR)
\begin{itemize}
\item Fundamentals of multiple imputation
\item Rubin's variance formula
\item Proper versus improper imputation
\item Asymptotic results
\item Imputation from a multivariate normal distribution 
\item Multivariate Imputation by Chained Equations (MICE)
\item Imputation using machine/deep learning
\item Additional results  
\end{itemize}

\item 03/10 - 04/02 -- Inverse Probability Weighted (IPW) Methods Under Missing At
  Random (MAR)
\begin{itemize}
\item IPW estimators for a single mean
\item IPW estimator in regression  
\item Weighted generalized estimating equations for longitudinal data
  with dropout
\item Inverse weighting at the occasion level
\item Inverse weighting at the individual level
\item Doubly robust augmented AIPW estimation
\item Inverse probability weighting with nonmonotone missingness
  \item Semiparametric theory
\end{itemize}

\item 04/07 - 04/16 - Nonignorable Missingness (MNAR)
\begin{itemize}
\item Selection models
\item Pattern mixture models
\item Shared parameter models
\end{itemize}

\item 04/16 - 04/23 -- Sensitivity Analysis to Deviations from Missing
At Random (MAR)
\begin{itemize}
\item Fundamental problem of identifiability
\item Estimation of a single mean
  \item Estimation of a single mean with auxiliary data
\item Longitudinal data with dropout
\end{itemize}

\item 04/28 -- Last day of class, Wrap-up, Presentations?
\end{description}

\pdfbookmark[1]{Further Resources}{s:resources}
\section*{Further Resources}

There is no textbook for this course; as noted above, we will use
lecture notes prepared by the instructor, and no other books are
required.  The notes cite publications where further information on
the specific developments presented can be found.  In addition, if you
are interested in more general, further reading on missing data
methods, the following books are good resources:
\begin{description}
\item Little, R. J. A. and Rubin, D. B. (2019). {\em Statistical
    Analysis With Missing Data, Third Edition}. New York: Wiley.

\item Molenberghs, G., Fitzmaurice, G., Kenward, M. G., Tsiatis,
  A. A., and Verbeke, G.  (2014).  {\em Handbook of Missing Data
    Methodology}.  Boca Raton, Florida: Chapman \& Hall/CRC Press.

\item Molenberghs, G. and Kenward, M. G. (2007). {\em MIssing Data in
    Clinical Studies}. Chichester, UK: Wiley.

\item O'Kelly, M. and Ratitch, B. (2014). {\em Clinical Trials with
    Missing Data: A Guide for Practitioners}. Chichester, UK: Wiley.

\item Schafer, J. L. (1997).  {\em Analysis of Incomplete Multivariate
    Data}.  London: Chapman \& Hall.

\item Tsiatis, A. A. (2006).  {\em Semiparametric Inference and
    Missing Data}.  New York: Springer. 
\end{description}

\pdfbookmark[1]{Administrative Matters}{s:admin}
\section*{Administrative Matters}

\bfem{Class Evaluations for NC State Students:} ClassEval is the
end-of-semester survey for students to evaluate the instruction of all
university classes. The current survey is administered online and
includes 12 closed-ended questions and 3 open-ended questions. Deans,
department heads, and instructors may add a limited number of their
own questions to these 15 common-core questions.

Each semester students’ responses are compiled into a ClassEval report
for every instructor and class. Instructors use the evaluations to
improve instruction and include them in their promotion and tenure
dossiers, while department heads use them in annual reviews. The
reports are included in instructors’ personnel files and are
considered confidential.

Online class evaluations will be available for students to complete
during the last two weeks of the semester for full-semester courses
and the last week of shorter sessions. Students will receive an email
directing them to a website to complete class evaluations. These
become unavailable at 8 am on the first day of finals.
\begin{itemize}
\item Contact ClassEval Help Desk: classeval@ncsu.edu
\item ClassEval website: \url{http://go.ncsu.edu/cesurvey}
\item More information about ClassEval:  \url{http://go.ncsu.edu/cesurvey}
\end{itemize}

\bfem{Academic integrity:} Students are required to comply with the
university policy on academic integrity found in the Code of Student
Conduct 11.35.01 sections 8 and 9; see
\url{http://policies.ncsu.edu/policy/pol-11-35-01}.  Therefore,
students are required to uphold the Pack Pledge: “I have neither iven
nor received unauthorized aid on this test or assignment.” Violations
of academic integrity will be handled in accordance with the Student
Discipline Procedures; see
\url{https://policies.ncsu.edu/regulation/reg-11-35-02/}.  Please
refer to the Academic Integrity web page,
\url{https://studentconduct.dasa.ncsu.edu/academic-integrity-overview/},
for a detailed explanation of the University’s policies on academic
integrity and some of the common understandings related to those
policies.

\bfem{Other Policies:}  Students are responsible for reviewing the NC State
University Policies, Rules, and Regulations that pertain to
their course rights and responsibilities, including:  
\begin{itemize}
\item Equal Opportunity and Non-Discrimination Policy Statement
  \url{https://policies.ncsu.edu/policy/pol-04-25-05} with additional
  references at \url{https://oied.ncsu.edu/equity/policies}

\item The Code of Student Conduct at
  \url{https://policies.ncsu.edu/policy/pol-11-35-01} and in
  particular the policy on academic integrity.  The instructor expects
  that students will abide by these policies.  As noted above,
  students {\em may} consult with one another on the homework, similar
  to how real researchers might consult with one another in carrying
  out their work.  However, students engaging in direct copying of the
  work or computer programs of fellow students will be considered in
  violation of policies on academic integrity.

\item Grades and Grade Point Average, regulation at
  \url{https://policies.ncsu.edu/regulation/reg-02-50-03}.
\end{itemize}

\bfem{Disability Resources:} Reasonable accommodations will be made
for students with verifiable disabilities.  For NC State students, to
take advantage of available accommodations, students must register
with Disability Resource Office (DRO),
\url{https://dro.dasa.ncsu.edu/}.  For more information on NC State's
policy on working with students with disabilities, please see the
Policies, Rules, and Regulations page maintained byt he DRO at
\url{https://dro.dasa.ncsu.edu/about-us/policies-rules-regulations/}
and the Academic Accommodations for Students with Disabilities Regulation
(REG02.20.01)
(\url{https://policies.ncsu.edu/regulation/reg-02-20-01/}).


\end{document}

